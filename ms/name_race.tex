\documentclass[12pt, letterpaper]{article}
\usepackage[titletoc,title]{appendix}
\usepackage{color}
\usepackage{booktabs}
\usepackage[tableposition=top]{caption}
\newcommand\fnote[1]{\captionsetup{font=small}\caption*{#1}}

\usepackage[usenames,dvipsnames,svgnames,table]{xcolor}
\definecolor{dark-red}{rgb}{0.75,0.10,0.10} 
\usepackage[margin=1in]{geometry}
\usepackage[linkcolor=dark-red,
            colorlinks=true,
            urlcolor=blue,
            pdfstartview={XYZ null null 1.00},
            pdfpagemode=UseNone,
            citecolor={dark-red},
            pdftitle={Predicting Race and Ethnicity From the Sequence of Characters in a Name}]{hyperref}
%\usepackage{multibib}
\usepackage{geometry} % see geometry.pdf on how to lay out the page. There's lots.
\geometry{letterpaper}               % This is 8.5x11 paper. Options are a4paper or a5paper or other... 
\usepackage{graphicx}                % Handles inclusion of major graphics formats and allows use of 
\usepackage{amsfonts,amssymb,amsbsy}
\usepackage{amsxtra}
\usepackage{natbib}
\DeclareRobustCommand{\firstsecond}[2]{#1}
\usepackage{verbatim}
\setcitestyle{round,semicolon,aysep={},yysep={;}}
\usepackage{setspace}             % Permits line spacing control. Options are \doublespacing, \onehalfspace
\usepackage{sectsty}             % Permits control of section header styles
\usepackage{lscape}
\usepackage{fancyhdr}             % Permits header customization. See header section below.
\usepackage{url}                 % Correctly formats URLs with the \url{} tag
\usepackage{fullpage}             %1-inch margins
\usepackage{multirow}
\usepackage{rotating}
\setlength{\parindent}{3em}
\usepackage{subcaption}

%\usepackage[T1]{fontenc}
%\usepackage{bm}
\usepackage{lmodern}
%\usepackage{libertine}
%\usepackage{gfsdidot}
\usepackage{chngcntr}

\title{\Large{Predicting Race and Ethnicity From the\\Sequence of Characters in a Name}\footnote{Data and scripts behind the analysis presented here can be downloaded from \url{http://github.com/appeler/ethnicolr_v2}. This paper is a new version of \citet{sood2018predicting}.}}

\author{Rajashekar Chintalapati\thanks{Rajashekar can be reached at \href{mailto:rajshekar.ch@gmail.com}\texttt{rajshekar.ch@gmail.com}} \and Gaurav Sood\thanks{Gaurav can be reached at \href{mailto:gsood07@gmail.com}\texttt{gsood07@gmail.com}}
\and Suriyan Laohaprapanon\thanks{Suriyan can be reached at: \href{mailto:suriyant@gmail.com}\texttt{suriyant@gmail.com}}}

\date{\vspace{.5cm}\normalsize{\today}}

\begin{document}
\maketitle

\begin{abstract}
To answer questions about racial inequality, we often need a way to infer race and ethnicity from a name. Until now, the bulk of the focus has been on optimally exploiting the last names list provided by the Census Bureau. But first names contain more information, especially for African Americans. To estimate the relationship between full names and race, we exploit Florida Voter Registration Data which includes self-reported race and ethnicity. We model the relationship between characters in a name and race and ethnicity using various techniques. We find that a model using Long Short-Term Memory works best with out-of-sample (OOS) precision and recall of .80 and .81 respectively. The equivalent numbers for the last name only model are .63 and .65. To illustrate the use of this method, we apply our method to campaign finance data to estimate the share of donations made by people of various racial groups and coverage of various races and ethnicities in the news.
\end{abstract}
\clearpage
\doublespace

How often are people of different races and ethnicities covered in the news? What percentage of campaign contributions come from African Americans? Are there racial gaps in healthcare delivery? Do minorities face discrimination in borrowing? To answer such questions, we often need a way to infer race and ethnicity from names. Given the important questions at stake, a number of researchers have worked on inferring race from names \citep[see, e.g.,][]{ambekar2009name, fiscella2006use, imai2016improving, rosenman2022race}. We contribute to the substantial literature.

\section*{Inferring Race and Ethnicity from Names}

\subsection*{Approaches} Researchers have used a variety of approaches to infer race and ethnicity from names. Some researchers have taken advantage of the list of popular last names provided by the Census Bureau \citep[see, e.g.,][]{fiscella2006use}. The approach suffers from two weaknesses---a biased small set of popular names, and a lack of first names. The information in the first name is especially vital for African Americans, whose last names are hard to distinguish from non-Hispanic whites, and whose first names tend to be distinctive \citep{bertrand2004emily}. Others have harvested Wikipedia data on names and their national origins and used crude features of names to build a national origins classifier \citep{ambekar2009name}. The Wikipedia data is small and suffers from a strong bias toward names of popular people. Still others have used private data and homophily to predict national origin \citep{ye2017nationality}. The limitation here is that the data are private. Still others like us use voter registration data \citep{sood2018predicting, parasurama2021racebert}. The voter registration data also has its own tradeoffs.\footnote{There is a parallel literature that combines voter registration data with census data to infer race where we have the name and location of a person \citep[see for e.g.,]{imai2016improving, kotovadeep}.} There is labeled data from only a few states, and not everyone is registered to vote.

\subsection*{Estimand} We predict the modal race and ethnicity of people with a particular name. In the SI, we show results from two more models. The first model predicts the probability distribution. The second model predicts all the most popular racial categories that make up at least 90\% of the people with the name.

\subsection*{Measurement of Race and Ethnicity} Florida Voter Registration data provide self-reported race and ethnicity of people. Even though race and ethnicity are self-reported, the limitations of the reporting instrument mean that the quality of the data is debatable. For one, Florida Voter Registration data treats race and ethnicity as one dimension with Hispanic treated as one category. Second, the instrument only allows crude categorization. For instance, Indian Americans are grouped under Asian and Pacific Islanders. 

\subsection*{Why Model?} If you picked a person at random with the last name Smith in the US in 2010 and were asked to guess the person's modal race (as measured by the Census), the best guess would be based on what is available from the aggregated Census file. It is the Bayes Optimal Solution. So what good are predictive models for? A few things---if you want to impute race and ethnicity for last names that are not in the census file, infer the race and ethnicity in different years than when the census was conducted (if some assumptions hold), infer the race of people in different countries (if some assumptions hold), etc. The biggest benefit comes in cases where both the first name and last name are known.

\section*{Data}
We exploit Florida Voting Registration data for the year 2022 \citep{sood_2017}. The Florida Voting Registration has information on nearly 15M voters along with their race. Given that we have very few people who identify as multi-racial and Native American, we condense them into Other. Our final dataset only has five categories: Asian/Pacific Islander, Hispanic, Non-Hispanic Blacks, Non-Hispanic Whites, and Other (see Table \ref{table:fl_data}). 

\begin{table}[h!]
\centering
\caption{Registered Voters in Florida by Race.}
\begin{tabular}{ l c }
\hline    
race & n \\
\hline
asian & 253,808 \\
hispanic & 2,179,106 \\
nh black & 1,853,690 \\
nh white & 8,757,268 \\
\hline
\end{tabular}
\label{table:fl_data}
\end{table}

To derive some baselines, we also use the Census Bureau last name data \citep{census2010}. The Census Bureau provides the frequency of all surnames occurring 100 or more times for the 2000 and 2010 census. Technical details of how the 2000 and 2010 data were collected can be found on the census website. 

We can use the Wikipedia data and the Florida Voting Registration data as is but the Census data needs to be transformed before being used. The dataset that the Census Bureau issues aggregates data for each last name and provides the percentage of people with the last name who are Black, White, Asian, Hispanic, etc. Given some names are more common than others (2,442,977 Americans had the last name Smith in 2010 according to the Census Bureau), and given our interest in modeling the population distribution, we take a weighted random sample from this data with weight equal to how common the last name is in the population. We assign the floor of \texttt{pctwhite} as proportion white, the floor of \texttt{pctblacks} as proportion black, etc. (Since we are using the floor, we lose a few observations but we ignore this drop off.) We use this as the final dataset.

\section*{Model and Validation}
To learn the association between the sequence of characters in names and race and ethnicity, we estimate an LSTM model \citep{graves2005framewise, gers1999learning} on approximately 1M randomly sampled names from the Florida Voter Registration Data and all the valid rows (n = 133,872) in the Wikipedia data. We estimate the last name model on a title case transformed version of the last name. For the full name model, we concatenate the last name and first name (ignoring the middle name) and again capitalize each word. We split the strings (last name or last name and first name) into two character chunks (bi-chars). For instance, Smith becomes {\tt Sm, mi, it, th}. Next, we remove infrequent bi-chars (occurring less than 3 times in the data) and very frequent bi-chars (occurring in over 30\% of the sequences in the data). We use the remaining bi-chars as our vocabulary. In the Florida Voting Registration Data, this leaves us with 1,146 bi-chars in the case of last name only data, and 1,604 bi-chars in the full name data. In the Wikipedia data, the corresponding numbers are 1,946 and 2,260. Next, we pad the sequences so that they are the same size. Finally, we use 20 as the window size for the last name only model and 25 for the full name model. 

On this set of sequences, we train a LSTM model using Keras \citep{chollet2015keras} and TensorFlow \citep{abadi2016tensorflow}. Before estimating the LSTM model, we embed each of the words onto a 32 length real-valued vector. We then estimate a LSTM with a .2 dropout and .2 recurrent dropout for regularization \citep{srivastava2014dropout}. The last layer is a dense layer with a softmax activation. Because it is a classification problem, we use log loss as the loss function. And we use ADAM for optimization \citep{kingma2014adam}. We fit the model for 15 epochs with a batch size of 32.

Table \ref{table:last_name_fl_voter_reg} presents some metrics that shed light on how well we did with the last name only model in predicting race OOS using the Florida Voter Registration Data. The OOS precision is .79, recall is .81, and f1-score, the harmonic mean of precision and recall, is .78. There is however sizable variation in recall across different racial and ethnic groups. For instance, recall is .95 for whites and just .21 for non-Hispanic blacks.\footnote{You see the same pattern when we estimate the model on the Census last name data. Recall for blacks on the model estimated on both the 2000 and 2010 Census last name data is .09 and .07 respectively (see Table \ref{table:last_name_census_2000} and Table \ref{table:last_name_census_2010}).}

\begin{table}[h!]
\centering
\caption{OOS Performance of the Last Name LSTM Model on the Florida Voter Registration Data.}
\begin{tabular}{ l c c c c }
\hline    
    race & precision & recall & f1-score & support \\
\hline
      asian &       0.77 &      0.41 &      0.54 &      4,527 \\
   hispanic &       0.74 &      0.70 &      0.72 &     18,440 \\
   nh black &       0.64 &      0.21 &      0.32 &     28,586 \\
   nh white &       0.82 &      0.95 &      0.88 &    146,009 \\

avg / total &       0.79 &      0.81 &      0.78 &    197,562 \\

\hline
\end{tabular}
\label{table:last_name_fl_voter_reg}
\end{table}

Compared to the last name only model, we do much better with a full name model. The OOS precision, recall, and f1-score for the full name model is .83, .84, and .83 respectively (see Table \ref{table:full_name_fl_voter_reg}). The gains are, however, asymmetric. Recall is considerably better for Asians and Non-Hispanic blacks with the full name---.49 and .43 respectively, compared to .41 and .21 respectively. The precision with which we predict non-Hispanic Blacks is also considerably higher---it is 9 points higher for the full name model. Given Asians and Hispanics have more distinctive last names, the improvement in precision in predicting both is smaller---negligible in the case of Asians and 2 points in the case of Hispanics.

\begin{table}[h!]
\centering
\caption{OOS Performance of the Full Name LSTM Model on the Florida Voter Registration Data.}
\begin{tabular}{ l c c c c }
\hline    
   race & precision & recall & f1-score & support\\
\hline
      asian &        0.77 &       0.49 &       0.60 &       4,527\\
   hispanic &        0.76 &       0.73 &       0.75 &      18,440\\
   nh black &        0.73 &       0.43 &       0.55 &      28,586\\
   nh white &        0.86 &       0.95 &       0.90 &     146,009\\

avg / total &        0.83 &       0.84 &       0.83 &     197,562\\
\hline
\end{tabular}
\label{table:full_name_fl_voter_reg}
\end{table}


\section*{Application}
To illustrate the utility of the models that we have developed here, we impute the race and ethnicity of individual campaign contributors in the 2000 and 2010 campaign contribution databases \citep{bonica2017database} using just the Census last name data and the Florida full name model. We then use the inferred race and ethnicity to estimate the proportion of total contributions made by people of different races. 

Based on the census last name data, in 2010, about 83.5\% of the contributions were made by Whites (see Table \ref{table:percentage_contrib_by_race}). But the commensurate number based on the Florida full name model was nearly 3\% more, 86.5\%. Moving to blacks, we see a similar story. Based on the census last name data, about 10.2\% of the contributed money came from blacks. But based on Florida full name model, the number is about 2.3\% lower, or a hefty 22.2\% relative change. The commensurate difference in estimated contributions by Hispanics is about 1\% or about 33\% relative change. Among Asians, the commensurate difference is about .5\% points or about 18\% relative change. A similar pattern holds for 2000. We see that the share of contributions made by Whites is smaller based on the Census last name data than the Florida full name model.

\begin{table}[h!]
\centering
\caption{Proportion of Total Amount Donated to Political Campaigns in 2000 and 2010 by People of Different Races/Ethnicities.}
\begin{tabular}{ l c c c c}
\hline
         & \multicolumn{2}{c}{Census} & \multicolumn{2}{c}{Florida}\\
\hline
race     &     2000     & 2010     &    2000    & 2010\\    
\hline
asian    &     2.22\%   & 2.74\%   &   2.00\%   & 2.28\%\\
black    &     11.04\%  & 10.22\%  &   8.93\%   & 7.92\%\\
hispanic &     3.24\%   & 4.32\%   &   3.23\%   & 3.31\%\\
white    &     83.49\%  & 82.71\%  &   85.84\%  & 86.49\%\\
\hline
\end{tabular}
\label{table:percentage_contrib_by_race}
\end{table}

\section*{Discussion}
We exploit a novel source of labeled data---voter registration files---along with the Wikipedia data to learn a model between sequences of characters in a name and race or ethnicity. Given poor African Americans tend to have distinctive first names, the biggest advantage in using the full name model is in our ability to detect African American names. We then use the model to infer the race of contributors in the DIME data and find that African Americans are less than a quarter percent of the donors. As we note, we also provide a Python package that exposes the models: \url{https://github.com/appeler/ethnicolr/}.

If you picked a random individual with last name Smith from the US in 2010 and asked us to guess this person's race (measured as crudely as by the census), the best guess would be based on what is available from the aggregated Census file. It is the Bayes optimal solution. So what good are last name only predictive models for? A few things. If you want to impute ethnicity at a more granular level, guess the race of people in different years (than when the census was conducted if some assumptions hold), guess the race of people in different countries (again if some assumptions hold), when names are slightly different (again with some assumptions), etc. The big benefit comes from when both the first name and last name is known. And there are a lot of important datasets, such as the campaign contributions dataset, the voter registration files of other states, news data, etc., where we have information on both the first and the last names. And we could make better predictions about the race and ethnicity by capitalizing on both the first and the last names, especially for African Americans, but also for other races and ethnicities.

The limitations of using the voter registration data are obvious. Not everyone is registered to vote, and blacks and Hispanics are especially likely not to be registered to vote \citep{ansolabehere2011gender}. If the names of those who are not on the voter registration file are systematically different from those who are, we are likely somewhat optimistic in our accuracy metrics. Another concern with using data from a single state is that the pattern of names in a single state are different from names given in other states. It is a very reasonable concern. We could overcome it by combining census last name models with state voter registration data models, but more research is needed to see how well we can do.

\clearpage
\bibliographystyle{apsr}
\bibliography{name}
\clearpage


\end{document}